\chapter{Полиномиальные вычисления}

\section{Вычисление значения полинома в точке}
Для вычисления значения функции в точке необходимо выполнить команду  
\comm{value}{(f, [var1, var2,\ldots, varn])}, 
где $f$~--- это полином,  в который на позиции переменных кольца подставляем соответствующие значения $var1, var2, \ldots, varn$. 

\underline{Пример. }

\vspace*{-2mm}
\begin{verbatim}
SPACE=R[x, y]; 
f=x^2+5x(y^3+x);
g=\value(f, [1, 2]); 
\print(g);
\end{verbatim}
%begindelete

\ex{$SPACE=R[x, y];$\\ 
\hspace*{4mm} $f=x^2+5x(y^3+x);$\\ 
\hspace*{4mm} $g=value(f, [1, 2]); $\\ 
\hspace*{4mm} $print(g);$}
{$g = 46. 00.$}
%enddelete

\section{Приведение полиномов к стандартному виду и разложение полиномов на множители}

Для приведения полинома к стандартному виду  необходимо выполнить команду  
\comm{expand}{(f)}, где $f$~--- это полином.

Для разложения полинома на множители необходимо выполнить команду  
\comm{factor}{(f)}, где $f$~--- это полином.


\underline{Пример. }

\vspace*{-2mm}
\begin{verbatim}
SPACE=Q[x, y]; 
f= (y^3+x)^2(x+1)^3;
 g=\expand(f);
h=\factor(g);
\print(g,h);
\end{verbatim}
%begindelete

\ex{$SPACE=Q[x, y]; $\\
\hspace*{4mm} $f= (y^3+x)^2(x+1)^3;$\\
\hspace*{4mm} $g= expand(f);$\\
\hspace*{4mm} $h= factor(g);$\\
\hspace*{4mm} $ print(g,h);$}{}
{$g = y^6x^3+3y^6x^2+3y^6x+y^6+2y^3x^4+6y^3x^3+6y^3x^2+2y^3x+x^5+3x^4+3x^3+x^2;$\\
\hspace*{4mm} $h=(x+1)^3(y^3+x)^2;$}

%enddelete

\section{Суммирование полинома по переменным.  Геометрические прогрессии }


Для суммирования полинома по переменным необходимо выполнить команду  
\comm{SumOfPol}{(f,  [x, y], [x1, x2, y1, y2])},  где
$f$~--- полином,  $x,  y$~--- переменные по которым ведется суммирование,  
$x1,  x2$~--- интервал суммирования по $x$,  
$y1,  y2$~--- интервал суммирования по $y$. 


Если интервалы суммирования для всех переменных совпадают, то можно записать  
\comm{SumOfPol}{(f,  [x, y], [x1, x2])},  где 
$x1,  x2$~--- интервал суммирования по $x$ и $y$. 

\underline{Пример.}

\vspace*{-2mm}
\begin{verbatim}
SPACE=R[x, y, z];
f=x^2z+xy+y^3xz;
res=\SumOfPol(f, [x, y], [2, 4, -2, 3]);
\print(res);
\end{verbatim}
%begindelete

\ex{$SPACE=R[x, y, z];$\\
\hspace*{4mm} $f=x^2z+xy+y^3xz;$\\
\hspace*{4mm} $res=SumOfPol(f, [x, y], [2, 4, -2, 3]);$\\
\hspace*{4mm} $print(res);$}
{$res = 417. 00z+27. 00.$}
%enddelete
 
Для преобразования полинома с помощью формулы суммы геометрической прогрессии необходимо выполнить команду  
\comm{SearchOfProgression}{(f)}. 
Данная команда ищет геометрическую прогрессию с наибольшим числом членов среди мономов полинома, затем делает это еще раз для оставшихся членов и так далее. Найденные прогрессии записываются в виде $S_n=b_1(q^n-1)/(q-1)$,  где 
$S_n$~--- сумма первых $n$ членов,  $b_1$~--- первый член геометрической прогрессии,  $q$~--- знаменатель прогрессии. 

\underline{Пример. }

\vspace*{-2mm}
\begin{verbatim}
SPACE=R[x, y, z];
f = x^3 + x^4 + x^5 + x^6 + x^7 + x^8 + x^9 + x^{10} + x^{11} + x^{12} +x^{13};
g = x + x^5 + x^9 + x^13 + xyz + 7x^2y^2z^2 + 7x^3y^3z^3 + 100xy + x + x^2 +x^3 + x^4;
f1 = \SearchOfProgression(f);
g1 = \SearchOfProgression(g);
\print(f1, g1);
\end{verbatim}
%begindelete

\ex{$SPACE=R[x, y, z];$\\
\hspace*{4mm} $f=x^3+x^4+x^5+x^6+x^7+x^8+x^9+x^10+x^11+x^12+x^13;$\\
\hspace*{4mm} $g=x+x^5+x^9+x^13+xyz+7x^2y^2z^2+7x^3y^3z^3+100xy+x+x^2+x^3+x^4;$\\
\hspace*{4mm} $f1=SearchOfProgression(f);$\\
\hspace*{4mm} $g1=SearchOfProgression(g);$\\
\hspace*{4mm} $print(f1, g1);$}
{$f1 = (x^{14}-x^{3})/(x-1); $\\
\hspace*{4mm} $g1 = ((100. 00yx+x^{13}+x^{9}+x)+(z^{4}y^{4}x^{4}-zyx)/(zyx-1)+(x^{6}-x)/(x-1)+6. 00z^{3}y^{3}x^{3}+6. 00z^{2}y^{2}x^{2}). $}
 %enddelete

\section{Вычисление базисов Гребнера}


Для вычисления базиса Гребнера полиномиального идеала $[p_{1}, p_{2}, \ldots, p_{N}]$ над рациональными числами можно воспользоваться командой \comm{groebnerB}{(p_{1}, p_{2}, \ldots, p_{N})} или командой \comm{groebner}{(p_{1}, p_{2}, \ldots, p_{N})}. 
Команда \comm{groebnerB}{()} вычисляет базиса Гребнера, используя алгоритм Бухбергера, а команда \comm{groebner}{()} использует матричный вариант алгоритма, предложенный Фужером.  
Используется обратное лексикографическое упорядочение переменных. Порядок на переменных определяется в команде SPACE.

\underline{Примеры. }

\vspace*{-2mm}
\begin{verbatim}
SPACE = Q[x, y, z]; 
b = \groebnerB(x^4y^3 + 2xy^2 + 3x + 1, x^3y^2 + x^2, x^4y + z^2 + xy^4 + 3);  
\print(b);
\end{verbatim}

%begindelete
\ex{$SPACE=Q[x, y, z];$\\ 
\hspace*{4mm} $b=groebnerB(x^4y^3+2x y^2+3x+1,  x^3y^2+x^2,  x^4y+z^2+x y^4+3); $\\ 
\hspace*{4mm} $print(b);$}
{$b = [z^2-x^4+3x^2+(-10)x+9, y+(-9)x^4+(-3)x^3-x^2+(-81)x+27, x^5+9x^2+(-6)x+1];$}
%enddelete

\begin{verbatim}
SPACE = Z[x, y, z];
b = \groebner(x^4y^3 + 2xy^2 + 3x + 1, x^3y^2 + x^2, x^4y + z^2 + xy^4 + 3);  
\print(b);
\end{verbatim}

%begindelete
\ex{$SPACE=Q[x, y, z];$\\ 
\hspace*{4mm} $b=groebner(x^4y^3+2xy^2+3x+1,  x^3y^2+x^2,  x^4y+z^2+x y^4+3);$\\ 
\hspace*{4mm} $print(b);$
}
{$b = [z^2-x^4+3x^2+(-10)x+9, x^5+9x^2+(-6)x+1, y+(-9)x^4+(-3)x^3-x^2+(-81)x+27]. $}
%enddelete

\section{Вычисления в факторкольце по идеалу}

Функция \comm{reduceByGB}{(f, [g_1, \ldots, g_N])} редуцирует полином
$p$ с помощью данного множества полиномов $g_1, \ldots, g_N$.

\begin{verbatim}
SPACE = Q[x, y, z];
p = \reduceByGB(5y^2 + 3x^2 + z^2, [y + x, 5z^2 + 5z]);
\end{verbatim}

%begindelete
\ex{$SPACE = Q[x, y, z];$\\
\hspace*{4mm} $p = \backslash reduceByGB(5y^2 + 3x^2 + z^2, [y + x, 5z^2 + 5z]);$}
{$-z+8x^2;$}
%enddelete

В случае, когда второй аргумент не является редуцированным базисом
Гребнера, результат зависит от расположения полиномов в массиве: 
при наличии нескольких потенциальных редукторов выбирается первый из них.

\begin{verbatim}
SPACE = Q[x, y];
NotGB1 = [x + y, x^2 + y^2];
imForNotGBset1 = \reduceByGB(x^2 + y^2, NotGB1);
NotGB2 = [x^2 + y^2, x + y];
imForNotGBset2 = \reduceByGB(x^2 + y^2, NotGB2);
GB = \groebner(x+y, x^2+y^2);  
imForGB  = \reduceByGB(x^2 + y^2, GB);
\print(GB, imForNotGBset1, imForNotGBset2, imForGB);
\end{verbatim}

%begindelete
\ex{
$SPACE=Q[x,y];$\\
$NotGB1=[x+y,x2+y2];$\\
$imForNotGBset1=reduceByGB(x2+y2,NotGB1);$\\
$NotGB2=[x2+y2,x+y];$\\
$imForNotGBset2=reduceByGB(x2+y2,NotGB2);$\\
$GB=groebner(x+y,x2+y2);$\\
$imForGB =reduceByGB(x2+y2,GB);$\\
$print(GB,imForNotGBset1,imForNotGBset2,imForGB);$
} {
$GB = [y+x, x^2];$\\
$imForNotGBset1 = 2x^2;$\\
$imForNotGBset2 = 0;$\\
$imForGB = 0;$
}
%enddelete

\section{Решение систем нелинейных алгебраических уравнений}

Для решения системы нелинейных алгребраических уравнений вида:

$\left\{\begin{array}{rcl} p_{1} & = & 0,\\ p_{2} & = & 0,\\ & \ldots\\ p_{N} & = & 0,\\ \end{array} \right. $

используется команда \comm{solveNAE}{(p_{1}, p_{2}, \ldots, p_{N})}.

Перед нахождением корней вычисляется базис Гребнера системы.
Если базис содержит уравнения от одной переменной, они решаются, и корни подставляются в оставшиеся уравнения. Корни вычисляются численно.
Ответом является вектор решений, в котором каждый элемент в свою очередь является вектором с элементами, соответствующими одному решению. Переменные
в решении перечисляются в том же порядке, в котором они указаны при объявлении SPACE.

\begin{verbatim}
SPACE = R[x, y];
\solveNAE(x^2 + y^2 - 4, y - x^2);
\end{verbatim}

\begin{verbatim}
SPACE = R[a, b, c];
S = \solveNAE(a + b + c, a b + a c + b c, a b c - 1);
\end{verbatim}

%begindelete
\section{Контрольные задания}
В  Mathpar вычислите 
\begin{itemize}
 \item $f(1)+f(2)+f(3)+f(4)+f(5)$ для $f=-3x^3-x^2+x+2$,
 \item базис Гребнера полиномиального идеала для полиномов 
$x^2+xy$,  $4xy^3-2xy-4$,  
$y^2-x$,  $x^2y^2+x+y-6$. 
 \end{itemize}
%enddelete
