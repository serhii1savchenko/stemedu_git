\chapter{Операторы управления.  Процедурное программирование}

\section{Процедуры и функции}
Система Mathpar позволяет создавать свои процедуры и функции.  Для этого используется команда \comm{procedure}{}.  После команды указывается имя процедуры и в фигурных скобках описывается сама процедура. 

\smallskip

\underline{Пример.}

\vspace*{-3mm}
 
\begin{verbatim}
\procedure myProc2() {
  d = 4;
  \print(d);
}
\procedure myProc(c,  d) {
  if (c < d) {
    \return d;
  } else {
    \return d+5;
  }
}
\myProc2();
a = 10;
c = \myProc(5 + a, a);
\print(a, c);
\end{verbatim}
%begindelete

 Результат выполнения: \\
$d = 4;\  
a = 10;\  
c = 15.$ 
%enddelete

\section{Операторы ветвления и циклов}

 Система Mathpar дает возможность использовать операторы ветвления и циклов. 

\comm{if }{()} \{ \} \comm{else}{} \{ \}~--- оператор ветвления; 

\comm{while }{()} \{ \}~--- оператор цикла с предусловием; 

\comm{for }{(\ ;\ ;\ )} \{ \}~--- оператор цикла с счетчиком.  

\smallskip

\underline{Примеры. }

\vspace*{-3mm}

\begin{verbatim}
a = 5; b = 1;
if (b < a) {
  b = b + a;
} else {
  \print(a, b);
}
if (b < a) {
  b = b + a;
} else {
  \print(a, b);
}
\end{verbatim}
%begindelete

Результат выполнения:\\
$a=5;\ b=6;$

%enddelete
\begin{verbatim}
a = 0;
b = 10;
while (a < b) {
  a = a + 5;
  \print(a);
}
\end{verbatim}
%begindelete

Результат выполнения:\\
$a = 5;$
$a = 10;$

%enddelete
\begin{verbatim}
for (i = 3; i \le 11; i = i + 5) { 
  \print(i);
}
\end{verbatim}

%begindelete
Результат выполнения:\\
$i = 3;$ 
$i = 8.$

\section{Контрольные задания}
В  Mathpar напишите программу:

\begin{itemize}
  \item для поиска наибольшего коэффициента матрицы, 
  \item для вывода всех чисел от 1 до 3000,  которые делятся на 252,  а при делении на 101 дают в остатке 3. 
\end{itemize}
%enddelete
