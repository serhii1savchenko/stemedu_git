\chapter{Примеры решения задач по физике}
\section{Передача тепла}

\begin{verbatim}
"ЗАДАЧА 1"
"Кусок льда массой"
M = 10 кг;
"помещен в сосуд. Температура льда"
T = -10  \degreeC ;
"Найдите массу воды в сосуде после того, как сосуду 
сообщили количество тепла равное"
q = 20000 кДж;
"Удельная теплоемкость воды"
c_v = 4.2 кДж/(кг \degreeC);
"Удельная теплоемкость льда"
c_i = 2.1 кДж/(кг \degreeC);
"Удельная теплота плавления льда"
r = 330 кДж/кг;
"Удельная теплота испарения воды"
\lambda = 2300 кДж/кг;
END
\end{verbatim}

\vspace*{3mm}

\begin{verbatim}
"РЕШЕНИЕ ЗАДАЧИ 1"
"Искомую массу воды обозначим через x."
SPACE = R64[x];
"Обозначим количество теплоты требуемое для нагревания льда до 0 градусов:"
q_1 = M c_i (0 - T);
"для плавления всего льда:"
q_2 = M r;
"для нагревания воды до ста градусов:"
q_3 = M c_v (100 \degreeC);
"для испарения части воды"
q_4 = (M - x)\lambda;
"Здесь мы обозначили через x массу оствшейся в сосуде воды."
"По условию задачи должно выполняться равенство,
решая которое найдем неизвестное x:"
mass  = \solve (q = q_1 + q_2 + q_3 + q_4);
mass=\value(mass);
\print(mass);
\end{verbatim}
\vspace*{-3mm}
 


\section{Кинематика} 

\begin{verbatim}
"ЗАДАЧА 2"
"Кинематическое уравнение движения точки по прямой (по оси x) 
имеет вид $x = c_1 + c_2 t + c_3 t^3$."
"Найдите: (1) координату точки, (2) мгновенную скорость,
(3) мгновенное ускорение" 
END
\end{verbatim}\vspace*{-3mm}
 
 \
 
\begin{verbatim}
"РЕШЕНИЕ ЗАДАЧИ 2."
"Выбираем пространство с переменными $t, c_1, c_2, c_3$:"
SPACE = R64[t, c_1, c_2, c_3];
"Уравнение движения точки"
x = c_1 + c_2  t + c_3 t^3;
"Вычислим мгновенную срость"
v = \D_t(x);
"Вычислим мгновенное ускорение"
a = \D_t(v);
\print(x, v, a);
\end{verbatim}\vspace*{-3mm}

\

\begin{verbatim}
"ЗАДАЧА 2А"
"Решите предыдущую задачу, при условии, что  "
"коэффициенты c1, c2, c3 в уравнении имеют следующие значения"
Coeff = [4, 2, -0.5];
"и момент времени равен "
t_0 = 2 "секунды." 
END
\end{verbatim}\vspace*{-3mm}

\

\begin{verbatim}
"РЕШЕНИЕ ЗАДАЧИ 2А"
"Введем обозначение для элементов вектора  Coeff:"
cf=\elementOf(Coeff);
"Найдем числовое значение каждой функции (x, v, a)
в точке"
arg = [t_0, cf_{1}, cf_{2}, cf_{3}];
"(1) координата точки в момент времени $t_0$:"
x_0 = \value (x, arg);
"(2) мгновенная скорость точки в момент времени $t_0$:"
v_0 = \value (v, arg);
"(3) мгновенное ускорение точки в момент времени $t_0$:"
a_0 = \value (a, arg);
\print(x_0, v_0, a_0);
\end{verbatim}\vspace*{-3mm}

\

\section{Молекулярная физика} 
 
 \
 
\begin{verbatim}
"ЗАДАЧА 3"
"В центре горизонтальной трубки расположен столбик ртути длиной h" 
"Часть воздуха была выкачана и концы трубки запаяны.   
Трубка имеет длину l" 
"Когда трубка была поставлена вертикально, столбик ртути переместился вниз на расстояние $l_d$." 
"Ускорение свободного падения обозначим $g$, плотность ртути — $\rho$"
"Какое начальное давление было в трубке?"
END
\end{verbatim}\vspace*{-3mm}

\

\begin{verbatim}
"РЕШЕНИЕ ЗАДАЧИ 3"
"Пусть в трубке было начальное давление $p_0$. Введем пространство с переменной $p_0$:"
SPACE=R64[p_0];
"После поворота трубки давление в нижней части трубки повысилось, 
так как добавилось давление столбика ртути. Следовательно, новое давление стало равно:" 
p_1 = p_0+\rho g h; 
"Пусть s это площадь поперечного сечения трубки. 
Тогда начальный обьем нижней части трубки равен:"
v_0=  (l/2-h/2) s; 
" После поворота трубки обьем воздуха в нижней части трубки будет  равен:"
v_1=  (l/2-h/2-l_d) s;
"В соответствии с законом Бойля-Мариотта запишем и решим 
уравнение относительно неизвестной $p_0$:"
initialPressure = \solve(p_0 v_0=p_1 v_1 );
\print(initialPressure );
\end{verbatim}

\

\begin{verbatim}
"ЗАДАЧА 3А."
"Решите предыдущую задачу предполагая, что переменные 
имеют следующие числовые значения: "
h = 0.20 m;
l = 1 m;
l_d = 0.10 m;
g = 9.8 m/s^2;
\rho = 13600 kg/m^3;
END
\end{verbatim}

\

\begin{verbatim}
"РЕШЕНИЕ ЗАДАЧИ 3А"
p_1 = p_0 + \rho g h;
v_0 = (l/2 - h/2) S; 
v_1 = (l/2 - h/2 - l_d) S;
\solve(p_0 v_0 = p_1 v_1);
\end{verbatim}
