\chapter{Выбор окружения для математических объектов}

\section{Окружение}
  Прежде чем будет задан любой математический объект,  число,  функция или символ,  должно быть ясно определено <<окружение>>~--- пространство,  
в котором будут определяться объекты.  
В этой главе описываются способы задания окружения.  Перемещение из некоторого окружения в текущее,  как правило,  должно выполнятся явно,  с помощью функции {\bf toNewRing}.  В некоторых случаях такое преобразование к текущему окружению происходит автоматически.  

Для выбора окружения задается {\it алгебраическое пространство переменных}.  Оно
определяется именами переменных и числовыми пространствами,  в которых эти
переменные принимают значения.  Порядок переменных в списке переменных задает линейный порядок на этих переменных.  Слева направо располагаются переменные,  упорядоченные по старшинству от младших к старшим. 

По умолчанию определено пространство  $\mathbb{R}64[x,y,z,t]$ четырех переменных,   самая младшая~--- $x$,  самая старшая~--- $t$. 

В любой момент пользователь может сменить окружение,  задав новое алгебраическое пространство переменных с помощью команды установки <<SPACE=>>.  Например,  для задач вычислительной математики может быть достаточно пространства типа $\mathbb{R}64[x]$ или  $\mathbb{Q}[x]$.  Команда установки: <<SPACE=R64[x];>> или <<SPACE=Q[x];>>, соответственно. 

Если имя переменной начинается с символа $\backslash$ и заглавной буквы (верхний регистр), 
то такая переменная обозначает элемент алгебры, у которой {\it операция умножения некоммутативная},  для всех остальных переменных {\it операция умножения коммутативная}. 





\section{Числовые множества}
Определены следующие числовые множества:

Z --- множество целых чисел ${\mathbb Z}$, 

Zp --- конечное поле из p=MOD элементов  ${\mathbb Z}/p{\mathbb Z}$,  MOD~--- постоянная,  

Zp32 --- конечное поле из p=MOD32 элементов ${\mathbb Z}/p{\mathbb Z}$, MOD32 меньше $2^{31}$, 

Z64 --- кольцо целых чисел $z$ таких,  что $-2^{63} \leqslant z < 2^{63}$,  

Q --- множество рациональных чисел, 
 
R --- множество чисел с плавающей точкой для хранения приближенных действительных чисел с произвольной мантиссой,  

R64 --- множество чисел с плавающей точкой для хранения приближенных действительных чисел с двойной точностью (со стандартной 52-разрядной мантиссой и отдельным 11-разрядным полем для хранения порядка),  
 
R128 --- стандартные 64-битные числа с плавающей точкой для хранения приближенных действительных чисел со стандартной 52-разрядной мантиссой и отдельным 64-разрядным полем для хранения порядка,  

C --- комплексный класс,  образованный из  класса R,  

C64 --- комплексный класс,  образованный из  класса  R64,  

C128 --- комплексный класс,  образованный из  класса  R128, 

CZ --- комплексный класс,  образованный из  класса  Z,  

CZp --- комплексный класс,  образованный из  класса  Zp,  

CZp32 --- комплексный класс,  образованный из  класса  Zp32,  

CZ64 --- комплексный класс,  образованный из  класса  Z64,  

CQ --- комплексный класс,  образованный из  класса  Q.  

Примеры простых полиномиальных колец: 

SPACE = Z [x,  y,  z]; 

SPACE = R64 [u,  v];  

SPACE = C [x]. 

\section{Определение нескольких числовых множеств}


Разрешается устанавливать
алгебраические пространства из нескольких числовых множеств,  например,  пространство <<C[z]R[x, y]Z[n, m]>> позволяет
работать с пятью именами переменных,  определенных в множествах $\mathbb{C}$, $\mathbb{R}$ и $\mathbb{Z}$,  соответственно. 
Первое множество считается основным и к нему будут приводится,  при необходимости, 
все остальные переменные.  В данном случае это $\mathbb{C}$. 


Его можно рассматривать как кольцо полиномов пяти переменных над $\mathbb{C}$,  при этом оно обладает дополнительными
свойствами.  Если полином не содержит переменной $z$,  то это полином с коэффициентами из $\mathbb{R}$.  Если
полином не содержит переменных $z$, $x$, $y$,  то это полином с коэффициентами из $\mathbb{Z}$. 


Примеры: 

SPACE=Z[x, y]Z[u]; 

SPACE=R64[u, v]Z[a, b]; 

SPACE=C[x]R[y, z]; 

 

 Кольцо <<Z[x, y, z]Z[u, v, w]>>,  в котором шесть переменных разделены
на две группы,  можно использовать для задач в которых строятся полиномы,  у которых коэффициенты
являются полиномами или функциями других переменных. 
Например,  характеристический полином для матрицы над кольцом $\mathbb{Z}[x,\ y,\ z]$ будет получен как полином
с неизвестной $u$,  коэффициенты которого лежат в кольце $\mathbb{Z}[x,\ y,\ z]$. 


% \section{Групповые алгебры}

 
%Групповую алгебру обозначает символ $G$.  После него стоит список образующих,  а перед ним~---
%% пространство, в котором действует группа. Образующие группы некоммутативны.

%Примеры свободных групповых алгебр: 

%SPACE=Z[x, y]G[U, V]; (образующие U, V),  

%SPACE=R64[u, v]G[A, B]; (образующие A, B), 

%SPACE=C[]G[X, Y, Z, T]; (образующие X, Y, Z, T). 



%Каждый элемент алгебры является суммой термов с коэффициентами,  которые являются функциями. 
%Например,  <<R64[x, y]G[X, Y, Z]>>~--- это свободная групповая алгебра с тремя некоммутативными
 %образующими X, Y, Z над функциями
%в  $\mathbb{R}64[x,\ y] $.  Тогда,  например,   
% $A=(t^2+1)X + \sin(t)Y + 3X^2y^3 +(t^2+1)XY^3X^2Y^{-2}x^2$~--- элемент такой алгебры. 



 
 \section{Идемпотентные алгебры.  Тропическая математика. }
 
 Кроме классических числовых алгебр с операциями <<+,~-,~*>> и операцией <</>> для полей,  будут доступны 
 пользователю и идемпотентные алгебры.  Для числового множества $\mathbb{R}64$,  можно будет использовать алгебры $R64MaxPlus$,  $R64MinPlus$,  $R64MaxMin$,  $R64MinMax$,  $R64MaxMult$,  $R64MinMult$.  Для числового множества $\mathbb{R}$,  можно будет использовать алгебры $RMaxPlus$,  $RMinPlus$,  $RMaxMin$,  $R64MinMax$,  $RMaxMult$,  $RMinMult$.  Для числового множества $\mathbb{Z}$,  можно будет использовать алгебры $ZMaxPlus$,  $ZMinPlus$, $ZMaxMin$,  $ZMinMax$,  $ZMaxMult$,  $ZMinMult$.  

%begindelete
\smallskip

\underline{Пример. }

\vspace*{-3mm}
%enddelete
\begin{verbatim}
SPACE=ZMaxPlus[x, y];
a=2; b=9; c=a+b; d=a*b; \print(c, d)
\end{verbatim}
%begindelete

Результат выполнения:\\
$c = 9; $\\
$d = 11.$
%enddelete

\section{Константы}
Можно установить или заменить следующие постоянные. 

{\bf FLOATPOS} --- число десятичных знаков после запятой,  которые выводятся на печать. 
По умолчанию принимается значение 2. 

{\bf MachineEpsilonR} --- машинное эпсилон для чисел типа  R. По умолчанию принимается значение $10^{-29}$. 
Число, модуль которого меньше $10^{-29}$, считаестя машинным нулем.  
Для установки нового значения $10^{-30}$ нужно ввести команду <<MachineEpsilonR64=30>>.
 

MachineEpsilonR64 --- машинное эпсилон для чисел типа  R64. По умолчанию принимается значение $2^{-36}$. 
Число, модуль которого меньше $2^{-36}$, считается машинным нулем. 
Отметим, что числа R64  имеет 52 разряда в мантиссе,
Для установки нового значения $2^{-48}$ нужно ввести команду <<MachineEpsilonR64=48>>.

Постоянная MachineEpsilonR (и MachineEpsilonR64) используется при факторизации полиномов
с коеффициентами типа R (или R64). Каждый коэффициент такого полинома будет 
предварительно делиться на число $MachineEpsilonR$ (или $MachineEpsilonR64$) и округляется до целого значения. 


{\bf ACCURACY} определяет число точных десятичных позиций после запятой для чисел типа $R$ и $C$ в 
операциях умножения и деления. По умолчанию ACCURACY имеет значение $MachineEpsilonR * 10^{-5}$.  
Если n<m, то команда <<MachineEpsilonR=n/m>> установит одновременно MachineEpsilonR=$10^{-n}$ 
и ACCURACY=$10^{-m}$.



{\bf MOD32} --- модуль для простого поля,  не превосходящий $2^{31}$ (по умолчанию принимается значение 268435399). Простое число MOD32 -- характеристика конечного поля.
Константа MOD32 используется в том случае, когда вычисления происходят в конечном поле 
Zp32 и она должна быть меньше числа $2^{31}$. 

{\bf MOD} --- модуль типа Z для простого поля (по умолчанию принимается значение 268435399).  Простое число MOD~--- это характеристика конечного поля, но в отличие от MOD32 у него нет ограничения на абсолютное значение. Константа MOD используется в том случае, когда вычисления происходят в конечном поле 
Zp.

{\bf RADIAN} может принимать значения 1 или 0. Если  RADIAN = 1, то углы измеряются в радианах, иначе --- в градусах. По умолчанию RADIAN=1.

{\bf STEPBYSTEP}  может принимать значения 1 или 0. Если  STEPBYSTEP = 1, то будут выводиться промежуточные результаты вычислений. По умолчанию STEPBYSTEP = 0.

{\bf EXPAND}  может принимать значения 1 или 0. Если EXPAND = 1, то во входном выражении будут раскрываться все скобки. По умолчанию EXPAND = 1.

{\bf SUBSTITUTION}  может принимать значения 1 или 0. Если SUBSTITUTION = 1, то во входном выражении будут подставляться вместо имен выражений их значения, если они были определены раньше. По умолчанию SUBSTITUTION = 1.


\smallskip

\underline{Пример. }

\vspace*{-3mm}

\begin{verbatim}
SPACE=Zp32[x, y]; 
MOD32=7; 
f1=37x+42y+55; 
f2=2f1;  
\print(f1, f2 );
\end{verbatim}
%begindelete

Результат выполнения:\\
$f1 = 2x-1; $ \\
$f2 = 4x+5. $
\section{Контрольные задания}
В Mathpar вычислите 
\begin{itemize}
 \item значение функции $f(x)=\sqrt{\sin ^2(5x-1)+e^x}$ в $\mathbb{R}$ при $x=7$, 
 \item значение функции $f(x)=x^3+10x$ в $\mathbb{Z}/(11)\mathbb{Z}$ при $x=7$. 
 \end{itemize}
%enddelete
