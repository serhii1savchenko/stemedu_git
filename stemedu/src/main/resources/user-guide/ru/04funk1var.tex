\chapter{Функции одной и нескольких переменных}

\section{Математические функции}
Приняты следующие обозначения для элементарных функций и констант. 

\subsection{Константы}
\hspace*{4mm}  $\backslash$i --- мнимая единица, 

 $\backslash$e --- основание натурального логарифма, 

 $\backslash$pi --- число $\pi$,  т. е.  отношение длины окружности к диаметру, 

 $\backslash$infty --- знак бесконечности. 


\subsection{Функции одного аргумента}

\hspace*{4mm} $\backslash$ln --- натуральный логарифм, 

 $\backslash$lg --- десятичный логарифм, 

 $\backslash$sin --- синус, 

 $\backslash$cos --- косинус, 

 $\backslash$tg --- тангенс, 

 $\backslash$ctg ---котангенс, 

 $\backslash$arcsin --- арксинус, 

 $\backslash$arccos --- арккосинус, 

 $\backslash$arctg --- арктангенс, 

 $\backslash$arcctg --- арккотангенс, 

 $\backslash$sh --- синус гиперболический, 

 $\backslash$ch --- косинус гиперболический, 

 $\backslash$th --- тангенс гиперболический, 

 $\backslash$cth --- котангенс гиперболический, 

 $\backslash$arcsh --- арксинус гиперболический, 

 $\backslash$arcch --- арккосинус гиперболический, 

 $\backslash$arcth --- арктангенс гиперболический, 

 $\backslash$arccth --- арккотангенс гиперболический, 

 $\backslash$exp --- экспонента, 

 $\backslash$sqrt --- корень квадратный, 

 $\backslash$abs --- абсолютное значение для действительных чисел,  модуль для комплексного числа;

 $\backslash$sign --- знак числа. Возвращает 1, 0,  -1,  когда число положительное, ноль или отрицательное  соответственно;

$\backslash$unitStep$(x,a)$ --- это функция, которая при $x \geqslant 0$ принимает значение $1$, а
при $x<0$ принимает значение $0$;

 $\backslash$fact   --- факториал.  Определен для целых положительных чисел.  Можно писать в привычном виде,  например, <<5!>>. 

\subsection{Функции двух аргументов}



\hspace*{4mm} $\widehat{\ }{}$  --- степень;

 $\backslash$log --- логарифм от функции по указанному основанию;

 $\backslash$rootOf(x, n) --- корень степени n из x; 

 $\backslash$Gamma --- функция Гамма;

 $\backslash$Gamma2 --- функция Гамма 2;

 $\backslash$binomial --- число сочетаний. 

%begindelete
\smallskip

\underline{Примеры. }
%enddelete

\begin{verbatim}
SPACE = R64[x, y];
f1 = \sin(x);
f2 = \sin(\cos(x + \tg(y)));
f3 = \sin(x^2) + y;
\print(f1, f2, f3);
\end{verbatim}\vspace*{-3mm}
%begindelete

Результат выполнения:\\
\hspace*{4mm} $f1 = sin(x); $\\
\hspace*{4mm} $f2 = sin(cos(x+tg(y))); $\\
\hspace*{4mm} $f3 = sin(x^{2})+y. $
%enddelete

\section{Вычисление значений функции в точке}

Для вычисления значения функции в точке необходимо выполнить команду 
\comm{value}{(f, [var1, var2,\ldots, varn])}, 
где $f$~--- функция,  а $var1,\  var2,\ldots,\ varn$~--- значения соответствующих переменных. 

Для тригонометрических функций мерой угла считается радиан или градус.  Указание меры угла определяется константой RADIAN.  
Если не указывать угловую меру,  то угловой мерой выбирается радиан.  Чтобы поменять угловую меру с радиан на градусы, 
нужно выполнить команду <<RADIAN=0;>>.  Если же нужно поменять угловую меру с градусов на радианы,  то
нужно выполнить команду <<RADIAN=1;>>. 

Если аргументами являются целые числа $15k$ и $18k$  градусов или
  $\pi k/12$ и $\pi k/10$ радиан $(k\in \mathbb Z)$,  то
значениями тригонометрических функций являются алгебраические числа.  
%begindelete
\smallskip

\underline{Примеры. }

\vspace*{-2mm}%enddelete
\begin{verbatim}
SPACE = R[x, y];
f = \sin(x^2 + \tg(y^3 + x));
g = \value(f, [1, 2]);
\print(g);
\end{verbatim}
\vspace*{-2mm}
%begindelete

\ex{ $SPACE=R[x, y];$\\ 
\hspace*{4mm} $f=sin(x^2+tg(y^3+x));$\\ 
\hspace*{4mm} $g=value(f,\ [1,\ 2]); $
\hspace*{4mm} $print(g);$}{$g = 0. 52;$}
%enddelete

\begin{verbatim}
SPACE = Z[x];
RADIAN = 0;
f = \sin(x); 
g = \value(f, 15); 
\print(g);
\end{verbatim}
\vspace*{-2mm}
%begindelete

\ex{$SPACE=Z[x];$\\ 
\hspace*{4mm} $RADIAN=0; $\\
\hspace*{4mm} $f=sin(x); $\\
\hspace*{4mm} $g=value(f, 15); $\\
\hspace*{4mm} $print(g);$}{$g = (\sqrt{6}-(\sqrt{2}))/(4);$}
%enddelete

\begin{verbatim}
SPACE = Z[x];
RADIAN = 0;
f = \sin(x);
g = \value(f, 225);
\print(g);
\end{verbatim}
%begindelete
\vspace*{-2mm}


Результат выполнения:\\
$g = (-1\cdot \sqrt{2})/(2);$
%enddelete

\begin{verbatim}
SPACE = Z[x];
RADIAN = 0;
f = \cos(x);
g = \value(f, 54);
\print(g);
\end{verbatim}
%begindelete

Результат выполнения:\\
$g = \sqrt{(5-\sqrt{5})/(8)};$
%enddelete

\begin{verbatim}
SPACE = Z[x];
RADIAN = 0;
f = \tg(x);
g = \value(f, 126);
\print(g);
\end{verbatim}
%begindelete
\vspace*{-2mm}


Результат выполнения: \\
$g = (-1\cdot \sqrt{(1+2\cdot \sqrt{5}/(5))});$
%enddelete

\begin{verbatim}
SPACE = Z[x];
RADIAN = 0;
f = \sin(x);
g = \value(f, 216);
\print(g);
\end{verbatim}
%begindelete

Результат выполнения: \\
$g = (-1\cdot \sqrt{(5-\sqrt{5})/(8)});$
%enddelete

\begin{verbatim}
SPACE = Z[x];
RADIAN = 0;
f = \cos(x);
g = \value(f, 108);
\print(g);
\end{verbatim}
%begindelete
\vspace*{-2mm}


Результат выполнения:\\
$g = (1-\sqrt{5})/(4). $
%enddelete

\section{Подстановка выражений в функции}

Для вычисления композиции функций нужно <<подставлять>> в функцию вместо ее аргументов другие функции.  Для этого
необходимо выполнить команду   \comm{value}{(f, [func1, func2,\ldots, funcn])},  где $f$~--- данная функция,  
$func1, func2,\ldots, funcn$~--- функции,  которые подставляются вместо соответствующих переменных. 
%begindelete
\smallskip

\underline{Пример. }

\vspace*{-2mm}%enddelete
\begin{verbatim}
SPACE = Z[x, y];
f = x + y;
g = f^2;
r = \value(g, [x^2, y^2]);
\print(r);
\end{verbatim}
%begindelete

\ex{$g = y^{2}+2yx+x^{2}; $ \\
\hspace*{4mm} $f = y+x;$}
{$r=y^{4}+2y^{2}x^{2}+x^{4}$.}
%enddelete

\section{Вычисление предела функции в точке }

Для вычисления предела функции в точке необходимо выполнить команду 
\comm{lim}{(f, var)},  
где $f$~--- это  функция,  а $var$~--- точка,  возможно бесконечная,  в которой требуется найти предел,  конечный или бесконечный. 

%begindelete
\smallskip
\underline{Примеры. }
%enddelete
\begin{verbatim}
SPACE = R64[x];
f = \sin(x) / x;
g = \lim(f, 0);
\print(g);
\end{verbatim}
%begindelete

\vspace*{-3mm}


\ex{$SPACE=R64[x];$ \\ 
\hspace*{4mm} $f=sin(x)/x; $ \\
\hspace*{4mm} $g=lim(f, 0);$ \\ 
\hspace*{4mm} $print(g);$}{
$g = 1. 00;$}
%enddelete

\begin{verbatim}
SPACE = R64[x];
f = (x^2 - 2x + 2) / (x^2 + x - 2);
g = \lim(f, 1);
\print(g);
\end{verbatim}
\vspace*{-3mm}
%begindelete

 \ex{$SPACE=R64[x];$ \\  
\hspace*{4mm} $f=(x^2-2x+2)/(x^2+x-2);$ \\   
\hspace*{4mm} $g=lim(f, 1);$ \\  
\hspace*{4mm} $print(g);$
}
{$g = \infty;$}
%enddelete

\begin{verbatim}
SPACE = R64[x];
f = \sin(x + 3) / (x^2 + 6x + 9);
g = \lim(f, -3);
\print(g);
\end{verbatim}
\vspace*{-3mm}
%begindelete

 \ex{$SPACE=R64[x]; $ \\
\hspace*{4mm} $f=\sin(x+3)/(x^2+6x+9); $ \\
\hspace*{4mm} $g=lim(f, -3); $ \\
\hspace*{4mm} $print(g);$
}{$g = \infty;$ }
%enddelete

\begin{verbatim}
SPACE = R64[x];
f = (1 + 1 / x)^x;
g=\lim(f, \infty);
\print(g);
\end{verbatim}
\vspace*{-3mm}
%begindelete

\ex{$SPACE=R64[x]; $ \\
\hspace*{4mm} $f=(1+1/x)^x; $ \\
\hspace*{4mm} $g=lim(f, \infty); $ \\
\hspace*{4mm} $print(g);$}
{$g = 2. 72. $}
%enddelete

\section{Дифференцирование функций}
Для вычисления производной функции f по переменной y из кольца $\mathbb{Z}[x, y, z]$
 необходимо выполнить команду \comm{D}{(f, y)}.  Вычисление третьей производной по $y$ можно выполнить
 $\backslash {\mathbf {D}} (f, [y\widehat{\ }{}3]) $. 
Если необходимо найти производную функции $f$ один раз
 по первой переменной из текущего кольца (в данном случае $x$) можно записать  \comm{D}{(f)} или \comm{D}{(f,x)} . 

 Для нахождения смешанной производной первого порядка от функции $f$ существует команда \comm{D}{(f, [x, y])},  для нахождения производной высших порядков нужно использовать команду  $\backslash {\mathbf {D}} (f, [x \widehat{\ }{} k, z \widehat{\ }{} m, y \widehat{\ }{} n])$,  где $k,  m,  n$ указывают,  какого порядка по соответствующей переменной вычисляется производная. 

%begindelete
\smallskip

\underline{Примеры. }

\vspace*{-3mm}%enddelete
\begin{verbatim}
SPACE=Z[x, y];
f = \sin(x^2 + \tg(y^3 + x));
h= \D(f, y);
\print(h);
\end{verbatim}
\vspace*{-3mm}
%begindelete

 Результат выполнения:\\
\ex{$SPACE=Z[x, y]; $ \\
\hspace*{4mm} $f=sin(x^2+ tg(y^3+x)); $ \\
\hspace*{4mm} $h= D(f, y);$ \\ 
\hspace*{4mm} $print(h);$}
{$h = 3y^2 cos(x^2+tg(y^3+x))/(cos(y^3+x))^2;$}
%enddelete

\begin{verbatim}
SPACE = Z[x, y];
f = \sin(x^2 + \tg(y^3 + x));
h = \D(f);
\print(h);
\end{verbatim}
\vspace*{-3mm}
%begindelete

\ex{$SPACE=Z[x, y]; $ \\
\hspace*{4mm} $f=sin(x^2+ tg(y^3+x)); $ \\
\hspace*{4mm} $h= D(f); $ \\
\hspace*{4mm} $print(h);$}
{$ h = (2x\cos(x^2+\tg(y^3+x))(\cos(y^3+x))^2+\cos(x^2+\tg(y^3+x)))/(\cos(y^3+x))^2;$}
%enddelete

\begin{verbatim}
SPACE = Z[x, y, z];
f = x^8y^4z^9;
g = \D(f, [x^2, y^2, z^2]);
\print(g);
\end{verbatim}
\vspace*{-3mm}
%begindelete

\ex{$SPACE=Z[x, y, z]; $ \\
\hspace*{4mm} $f=x^8y^4z^9; $ \\
\hspace*{4mm} $g=D(f, [x^2, y^2, z^2]);$ \\ 
\hspace*{4mm} $print(g);$}
{$g = 48384z^{7}y^{2}x^{6}. $ }
%enddelete

\section{Интегрирование композиций элементарных функций}

Символьное интегрирование композиций элементарных функций выполняется командой  
\comm{int}{(f(x))d x}.

%begindelete
\smallskip
\underline{Примеры. }
%enddelete

\begin{verbatim}
SPACE = Z[x, y, z];
l1 = \int(x^6yz + 3x^2y - 2z) d x;
dl1 = \D(l1,x);
l2 = \int(x^6yz + 3x^2y - 2z) d y;
dl2 = \D(l2,y);
l3 = \int(x^6yz + 3x^2y - 2z) d z;
dl3 = \D(l3,z);
\print(l1, dl1,l2, dl2,l3, dl3);
\end{verbatim}
\vspace*{-3mm}

%begindelete
\ex{$SPACE=Z[x, y, z]; $ \\
\hspace*{4mm} $l1=\int(x^6yz + 3x^2y - 2z)d x;$ \\
\hspace*{4mm} $dl1=D(l1,x); $ \\
\hspace*{4mm} $l2=\int(x^6yz + 3x^2y - 2z)d y;$ \\
\hspace*{4mm} $dl2=D(l2,y); $ \\
\hspace*{4mm} $l3=\int(x^6yz + 3x^2y - 2z)d z;$ \\
\hspace*{4mm} $dl3=D(l3,z); $ \\
\hspace*{4mm} $print(l1, dl1, l2, dl2, l3, dl3);$}
{
$l1 = (1/7)zyx^7-2zx+yx^3; $\\
\hspace*{4mm} $dl1 = x^6yz + 3x^2y - 2z. $
$l2 = (1/2)zy^2x^6-2zy+(3/2)y^2x^2; $\\
\hspace*{4mm} $dl2 = x^6yz + 3x^2y - 2z.$
$l3 = (1/2)z^2yx^6-z^2+3zyx^2; $\\
\hspace*{4mm} $dl3 = x^6yz + 3x^2y - 2z.$
}
%enddelete


\begin{verbatim}
SPACE = R[x];
l = \int(1/(x^2-5x+6)) d x;
dl = \D(l,x);
\print(l, dl);
\end{verbatim}
\vspace*{-3mm}

%begindelete
\ex{$SPACE=Q[x, y, z]; $ \\
\hspace*{4mm} $l=\int(1/(x^2-5x+6))d x;$ \\
\hspace*{4mm} $dl=D(l,x); $ \\
\hspace*{4mm} $print(l, dl);$}
{
$l = (\ln(x-3)-\ln(x-2)); $\\
\hspace*{4mm} $dl = (1/(x-3)-1/(x-2)). $
}
%enddelete

\begin{verbatim}
SPACE = Q[x];
l = \int(\exp(x)+\exp(-x)) d x;
dl = \D(l,x);
\print(l, dl);
\end{verbatim}
\vspace*{-3mm}

%begindelete
\ex{$SPACE=Q[x, y, z]; $ \\
\hspace*{4mm} $l=\int(\exp(x)+\exp(-x))d x;$ \\
\hspace*{4mm} $dl=D(l,x); $ \\
\hspace*{4mm} $print(l, dl);$}
{
$l = (\exp(x)-((\exp(x))^{-1})); $\\
\hspace*{4mm} $dl = (\exp(x)+\exp(-x)). $
}
%enddelete

\begin{verbatim}
SPACE = Q[x];
l = \int(x*\exp(x^2)) d x;
dl = \D(l,x);
\print(l, dl);
\end{verbatim}
\vspace*{-3mm}

%begindelete
\ex{$SPACE=Q[x, y, z]; $ \\
\hspace*{4mm} $l=\int(x*\exp(x^2))d x;$ \\
\hspace*{4mm} $dl=D(l,x); $ \\
\hspace*{4mm} $print(l, dl);$}
{
$l = (\exp(x^2)/2); $\\
\hspace*{4mm} $dl = (x*\exp(x^2)). $
}
%enddelete

\begin{verbatim}
SPACE = Q[x];
l = \int((x*\ln(x)*\exp(x)+\exp(x))/x) d x;
dl = \D(l,x);
\print(l, dl);
\end{verbatim}
\vspace*{-3mm}

%begindelete
\ex{$SPACE=Q[x, y, z]; $ \\
\hspace*{4mm} $l=\int((x*\ln(x)*\exp(x)+\exp(x))/x)d x;$ \\
\hspace*{4mm} $dl=D(l,x); $ \\
\hspace*{4mm} $print(l, dl);$}
{
$l = (\ln(x)*\exp(x)); $\\
\hspace*{4mm} $dl = ((x*\ln(x)*\exp(x)+\exp(x))/x). $
}
%enddelete


\begin{verbatim}
SPACE = R64[x];
l = \int((\ln(x+3)+\ln(x+2)+\ln(x+1))) d x;
dl = \D(l,x);
\print(l, dl);
\end{verbatim}
\vspace*{-3mm}

%begindelete
\ex{$SPACE=R64[x, y, z]; $ \\
\hspace*{4mm} $l=\int((\ln(x+3)+\ln(x+2)+\ln(x+1)))d x;$ \\
\hspace*{4mm} $dl=D(l,x); $ \\
\hspace*{4mm} $print(l, dl);$}
{
$l = (((x*\ln(x+3)+3.00*\ln(x+3)+x*\ln(x+2)+2.00*\ln(x+2)+x*\ln(x+1)+\ln(x+1))-3x)); $\\
\hspace*{4mm} $dl = ((\ln(x+3)+\ln(x+2)+\ln(x+1))). $
}
%enddelete

\begin{verbatim}
SPACE = Q[x];
l = \int((2x^2+1)^3) d x;
dl = \D(l,x);
m=\factor(dl);
\print(l, m);
\end{verbatim}
\vspace*{-3mm}

%begindelete
\ex{$SPACE=Q[x, y, z]; $ \\
\hspace*{4mm} $l=\int((2x^2+1)^3)d x;$ \\
\hspace*{4mm} $dl=D(l,x); $ \\
\hspace*{4mm} $m=factor(dl); $ \\
\hspace*{4mm} $print(l, m);$}
{$l = (8/7)x^7+(12/5)x^5+2x^3+x; $\\
\hspace*{4mm} $m = (2x^2+1)^3. $}
%enddelete



\section{Упрощение композиции функций}
Для разложения любой тригонометрической или логарифмической функции с помощью тождеств:\\
$sin(x)cos(y) \pm cos(x)sin(y) = sin(x \pm y),$ \\
$cos(x)cos(y) \pm sin(x)sin(y) = cos(x \mp y),$ \\
$sin^2(x) + cos^2(x) = 1,$ \\
$cos^2(x) - sin^2(x) = cos(2x),$ \\
$ln(a) + ln(b) = ln(ab),$ \\
$ln(a) - ln(b) = ln(\dfrac{a}{b}),$ \\
используется команда \comm{Expand}{(f(x))} 

%begindelete
\smallskip
\underline{Примеры. }
%enddelete
\begin{verbatim}
SPACE=Q[x, y, z]; 
g=\ln(x^2*4x); 
f=\Expand(g); 
\print(f);
\end{verbatim}\vspace*{-3mm}
%begindelete

\ex{$SPACE=Q[x, y, z]; $ \\
\hspace*{4mm} $g=\ln(x^2*4x); $ \\
\hspace*{4mm} $f=Expand(g); $ \\
\hspace*{4mm} $print(f);$}
{\hspace*{4mm} $f=\ln(x^2) + \ln(4x);$} 
%enddelete

\begin{verbatim}
SPACE=Q[x, y, z]; 
g=\sin(x^2+4x+2\pi); 
f=\Expand(g); 
\print(f);
\end{verbatim}\vspace*{-3mm}
%begindelete

\ex{$SPACE=Q[x, y, z]; $ \\
\hspace*{4mm} $g=\sin(x^2+4x+2\pi); $ \\
\hspace*{4mm} $f=Expand(g); $ \\
\hspace*{4mm} $print(f);$}
{\hspace*{4mm} $f=(\sin(x^2)(\cos(4x)\cos(2)-\sin(4x)\sin(2))+\cos(x^2)(\sin(4x)\cos(2)+\cos(4x)\sin(2)));$} 
%enddelete

\begin{verbatim}
SPACE=Q[x, y, z]; 
g=\cos(\sin(x)+\cos(y)); 
f=\Expand(g); 
\print(f);
\end{verbatim}\vspace*{-3mm}
%begindelete

\ex{$SPACE=Q[x, y, z]; $ \\
\hspace*{4mm} $g=\cos(\sin(x)+\cos(y)); $ \\
\hspace*{4mm} $f=Expand(g); $ \\
\hspace*{4mm} $print(f);$}
{\hspace*{4mm} $f=(\cos(\cos(y))\cos(\sin(x))-\sin(\cos(y))\sin(\sin(x)));$}
%enddelete

Для разложения на множители выражений при помощи описанных выше тригонометрических и логарифмических тождеств, а также следующих тождеств:\\
$ln(a)^k = k\cdot ln(a),$\\
$e^{iz} + e^{-iz} = 2\cos(z),$\\
$e^{iz} - e^{-iz} = 2i\sin(z),$\\
$\ln(1+iz) - \ln(1-iz) = 2i\cdot arctg(z),$\\
$\ln(1-iz) - \ln(1+iz) = 2i\cdot arcctg(z),$\\
$e^{z} + e^{-z} = 2ch(z)$\\
$e^{z} - e^{-z} = 2i\cdot sh(z),$\\
используется команда \comm{Factor}{(f(x))}.

%begindelete
\smallskip
\underline{Примеры. }

%enddelete
\begin{verbatim}
SPACE=Q[x, y, z]; 
g=\log_{2}(x)+\log_{2}(y)-\log_{2}(xz)+\lg(y)+\lg(y)-\lg(z); 
f=\Factor(g); 
\print(f);
\end{verbatim}
%begindelete

\ex{$SPACE=Q[x, y, z]; $ \\
\hspace*{4mm} $g=\log_{2}(x)+\log_{2}(y)-\log_{2}(xz)+\lg(y)+\lg(y)-\lg(z);$ \\
\hspace*{4mm} $f=Factor(g); $ \\
\hspace*{4mm} $print(f);$}
{\hspace*{4mm} $f=\lg(y^2/z)+\log_{2}(y/z);$} 
%enddelete

\begin{verbatim}
SPACE=Q[x, y, z]; 
g=16\sin(x/48)\cos(x/48)\cos(x/24)\cos(x/12)\cos(x/6); 
f=\Factor(g); 
\print(f);
\end{verbatim}\vspace*{-3mm}
%begindelete

\ex{$SPACE=Q[x, y, z]; $ \\
\hspace*{4mm} $g=16\sin(\frac{x}{48})\cos(\frac{x}{48})\cos(\frac{x}{24})\cos(\frac{x}{12})\cos(\frac{x}{6});$ \\
\hspace*{4mm} $f=Factor(g); $ \\
\hspace*{4mm} $print(f);$}
{\hspace*{4mm} $f=\sin(0.33x);$} 
%enddelete
 
\begin{verbatim}
SPACE=C64[x, y, z]; 
g=\ln(1-\ix) - \ln(1+\ix) + \e^(\ix) - 2\e^(-\ix) + \sin(x)^2 - \cos(x)^2; 
f=\Factor(g); 
\print(f);
\end{verbatim}\vspace*{-3mm}
%begindelete

\ex{$SPACE=C64[x, y, z]; $ \\
\hspace*{4mm} $g=\ln(1-i x) - \ln(1+i x) + \exp(i x) - 2\exp(-i x) + \sin(x)^2 - \cos(x)^2;$ \\
\hspace*{4mm} $f=Factor(g); $ \\
\hspace*{4mm} $print(f);$}
{\hspace*{4mm} $f=(-1.00\cos(2x))+2.00i(\sin(x))+(-1.00\exp(-i x))+(2.00i(\arcctg(x)));$} 
%enddelete

Комбинируя команды \comm{Factor}{(f(x))} и \comm{Expand}{(f(x))}, можно упрощать более сложные выражения.

\begin{verbatim}
SPACE=R64[x, y, z]; 
g=(\sin(x+y) + \sin(x-y))\cos(x) + (\sin(x+y) + \sin(x-y))\sin(y); 
f=\Expand(g);
u=\Factor(f);
\print(f,u);
\end{verbatim}\vspace*{-3mm}
%begindelete

\ex{$SPACE=R64[x, y, z]; $ \\
\hspace*{4mm} $g=(\sin(x+y) + \sin(x-y))\cos(x) + (\sin(x+y) + \sin(x-y))\sin(y);$ \\
\hspace*{4mm} $f=Expand(g); $ \\
\hspace*{4mm} $u=Factor(g); $ \\
\hspace*{4mm} $print(f,u);$}
{\hspace*{4mm} $f=2.00\cos(y)\sin(x)\cos(x)+2.00\sin(y)\cos(y)\sin(x);$\\
\hspace*{4mm} $u=\sin(x)\sin(2.00y)+\cos(y)\sin(2.00x);$} 
%enddelete
%begindelete





\section{Контрольные задания}
В  Mathpar 
\begin{itemize}
 \item подставьте в функцию $f(x)=\sqrt{\sin ^2(5x-1)+e^x}$ вместо $x$ выражение $x+y$,  а вместо $y$ --- $5$, 
 \item найдите  $\lim (x^3+10x)/x^2$  при $x\rightarrow 0$, 
 \item найдите производную функции $f(x)=\sqrt{\sin ^2(5x-1)+e^x}$, 
 \item найдите $\int_0^2(x^3+10x)dx$. 
 \end{itemize}
%enddelete
