\chapter{Operators of control. Procedural programming}

\section{Procedures and functions} 

Mathpar system lets you create your procedures and functions. 
To do this, use the command \comm{procedure}{}. After the command \comm{procedure}{}, you must specify the name of the procedure, 
and then in the curly brackets describes the procedure itself.
\smallskip

\underline{Example.}

\vspace*{-3mm}
 
\begin{verbatim}
\procedure myProc2() {
  d = 4;
  \print(d);
}
\procedure myProc(c, d) {
  if (c < d) {
    \return d;
  } else {
    \return d + 5;
  }
}
\myProc2();
a = 10;
c = \myProc(5 + a, a);
\print(a, c);
\end{verbatim}
%begindelete

Returns: \\
$d = 4;\  
a = 10;\  
c = 15.$ 
%enddelete

\section{Operators of branching and looping}

You can use the operators of branching and looping:

\comm{if }{()} \{ \} \comm{else}{} \{ \}~--- the operator of branching; 

\comm{while }{()} \{ \}~--- cycle operator with precondition; 

\comm{for }{(\ ;\ ;\ )} \{ \}~--- cycle operator with the counter.  

\smallskip

%begindelete
\underline{Examples:}
%enddelete

\vspace*{-3mm}

\begin{verbatim}
a = 5;
b = 1;
if(b < a) {
  b = b + a;
} else {
  \print(a, b);
}
if(b < a) {
  b = b + a;
} else {
  \print(a, b);
}
\end{verbatim}
%begindelete

Returns:\\
$ a=5;\ b=6; $
%enddelete

\begin{verbatim}
a = 0;
b = 10;
while(a < b) {
  a = a + 5;
  \print(a);
}
\end{verbatim}
%begindelete

Returns: \\
$a = 5;$
$a = 10;$

%enddelete
\begin{verbatim}
for (i = 3; i \le 11; i = i + 5) { 
  \print(i);
}
\end{verbatim}
%begindelete

Returns:\\ $i = 3;$ 
$i = 8. $
%enddelete
