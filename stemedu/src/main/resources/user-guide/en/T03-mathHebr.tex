
\chapter{ דוגמאות לפתרונות של בעיות מתמטיות }

\section{סידרה  חשבונית}
 
\begin{verbatim}
"TASK-Z:  תרגיל  סידרה  חשבונית.
$( n > 2) $   איברים  n   נחונה  סידרה  חשבונית  שיש  בה
 $a_1, a_2, a_3, … , a_{n-1}, a_n$
$d$  :  הפרש  הסידרה  הוא
מהסידרה  הנתונה  בנו  סידרה  חדשה
$  a_{2}^2 – a_{1}^2,  a_{3}^2 – a_{2}^2, …,  a_{n}^2 – a_{n-1}^2$
$2d^2$   הוכח כי  סידרה  חדשה  היא  סידרה  חשבונית  שהפרשה  הוא
$a_{2}^2 – a_{1}^2 = 64$  :    נחון   
$d$ , $n$   הבע  את  האיבר  האחרון  בסידרה  החדשה  באמצעות 
$ a_{n}^2 – a_{n-1}^2 = 192  ,   d > 1 $  :    נחון גם
$n$   מצא  את  תחום  הערכים  של "
 END
\end{verbatim}

\vspace*{-3mm}  

\begin{verbatim}
"פתרון"
a_{k+1} = a_k + d;
a_{k+2} = a_k + 2*d;
" $b_1, b_2, b_3, … , b_{n-1}$ :סידרה חדשה"
b_k = (a_{k+1})^2 - (a_k)^2;
b_{k+1} = (a_{k+2})^2- (a_{k+1})^2;
"$D$  :  הפרש  הסידרה  הוא"
D = b_{k+1} - b_k;
\print(b_k, b_{k+1}, D);
\end{verbatim}

\vspace*{-3mm} 


\begin{verbatim}
":נחון"
b_1 = (a_2)^2 - (a_1)^2 = 64;
b_{n-1} = b_1 + D * (n-2);
\print(b_1, b_{n-1});
\end{verbatim}

\vspace*{-3mm} 

\begin{verbatim}
"$b_{n-1}=192;   g=d^2>1;$" 
 g=\solve(64+2g n-4g = 192,[g]);
\end{verbatim}

\vspace*{-3mm} 

\begin{verbatim}
\solve(g>1,[n])
\end{verbatim}

\vspace*{-3mm} 

\section{בעיה גיאומטרית} 

\begin{verbatim}
"TASK  בעיה גיאומטרית"  
" $(AD = AC)$ משולש שווה  שוקיים  - $\Delta ADC$"
" $\DC$  נקודה  בצד  $B$"
"$DC = 3*BC,  AB = BC$ כך ש "
"$\Delta ADC$ מצה  את  גודל הזוויות של משולשא"
"$16\sqrt(3)   הוא$  $\Delta ADC$  נחון גם כי שטח המשולש"
"$כך  ש  AC$  נקודה בצד $T$"
"$AC  ניצב  BT$"
"$DT$ חשב  את "
 END
\end{verbatim}

\vspace*{-3mm} 

\begin{verbatim}
"ציור לתרגיל"
x1=\sqrt(3); x3=x1/3;
p1=\tablePlot([[0,-1,2,0,-1,1],
            [0, x1,0,0,x1,0]]);
p11=\pointsPlot([[0,-1,2, 1,1],
            [0, x1,0, 0,x3]],['A','D','C','T','B'],[0,0,1,1,1]);
p2=\tablePlot([[1, 1, 0 ],
               [0,x3, 0]]);
\showPlots([p1,p11,p2,p2],[' ',' ',' ','noAxes' ])
\end{verbatim}

\vspace*{-3mm} 

\begin{verbatim}
"פתרון"
":סימון"
AB = BC = x;
AD = AC = y;
DB = 2*x;
DC = 3*x;
" $: \Delta ABC$  המשולש"
"$AB^2 = BC^2 + AC^2 - 2*BC*AC*\cos(C);$"
"$x^2 = x^2 + y^2 - 2*x*y*\cos(C);$"
"$y^2 - 2xy\cos(C) = 0;    y - 2x\cos(C) = 0;   y = 2x\cos(C);$"
" $: \Delta ADC$  המשולש"
"$AD^2 = DC^2 + AC^2 - 2*DC*AC*\cos(C);$"
"$y^2 = 9x^2 + y^2 - 2*3x*y*\cos(C);$"
"$9x^2 - 6xy\cos(C) = 0;  9x - 6y\cos(C) = 0;  9x - 6*2x\cos(C)* \cos(C) = 0;$"
"$9x - 12x(\cos(C))^2 = 0;  (\cos(C))^2 = 9/12; \cos(C) = \sqrt(3)/2;$"
SPACE=Z[C]; sol=\solveTrig(\cos(C) = \sqrt(3)/2); 
SPACE=R64[x]; \print(sol);
\end{verbatim}

\vspace*{-3mm} 

\begin{verbatim}
C = 30 "$^\circ$ זווית"
D = C = 30 "$^\circ$ זווית"
A = 180 - (C + D) " זווית" 
\print(A,C,D)
\end{verbatim}


\vspace*{-3mm} 

\begin{verbatim}
"המשולש שווה שוקיים   $\Delta ABC$"
"$AC$  ניצב  $BT$"
 TC = AT= y/2  "   וכאן הוא תיכון "
 y = 2x\cos(C)  = 2x*\cos(\pi/6); 
"$AC$ נקודת אמצע בצד  $T$"
"$S(\Delta DCT) = S(\Delta ADC)/2 = 16*\sqrt(3)/2 = 8*\sqrt(3)$   תיכון לכן   $DT$"
"$S(\Delta DCT) = 1/2 * DC * TC * \sin(C) = 1/2 * 3x * y/2 * 1/2 = 1/8 * 3x * x*\sqrt(3) = 3/8 * x^2*\sqrt(3);$"
"$8*\sqrt(3) = 3/8*x^2*\sqrt(3);$"
"$x^2 = 64/3;$"  
 x = 8/\sqrt(3);
" $: \Delta DTC$  המשולש"
 DC = 3x; 
 DT = \value(\sqrt( DC^2 + TC^2 - 2*DC*TC*\cos(\pi/6)));
\end{verbatim}