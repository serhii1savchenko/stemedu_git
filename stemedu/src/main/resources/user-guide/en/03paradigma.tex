\chapter{Environment for mathematical objects}
\section{Setting of environment}

The definition of any mathematical object, a number or function, a matrix or symbol, involves the definition of some environment, that is, the space which contains this object. 
To select the environment you have to set the {\it algebraic structure}.  
This algebraic structure is defined by numeric sets, algebraic operations in these sets and variable names.

First of all any user have to set an environment in Mathpar.

By default, a space of the four real variables $\mathbb{R}64[x,y,z,t]$ is defined. This is ring of polynomials with   coefficients in the ring of real numbers, the youngest is the variable  $ x $, the eldest is the variable $ t $.

User can change the environment, setting a new algebraic structure. 
 

For example the spaces $\mathbb{R}64[x]$ or $\mathbb{Q}[x]$ may be suitable to solve many problems of computational mathematics. 
The installation  command should be the follow: <<SPACE=R64[x];>> or <<SPACE=Q[x];>>. 

 
Moving a mathematical object from the previous environment to the current environment, as a rule, should be performed explicitly, using the $ toNewRing()$ function.  In some cases, such a transformation to the current environment is automatic. 
 
All other names which are not listed as a variables can be chosen arbitrarily by the user for any mathematical object.   

For example
$$a=x+1, \ \  f= \backslash \backslash sin(x+y) - a.$$
We follow the rule.
  If the object name begins with the symbol $\backslash$ and a {\it   capital letter} such object is an element of a {\it noncommutative} algebra, else object is an element of a  {\it   commutative} algebra.

\section{Numerical sets with standard operations} 
Current version of the system supports the following numerical sets with standard operations.

Z --- the set of integers ${\mathbb Z}$, 

Zp --- a finite field ${\mathbb Z}/p{\mathbb Z}$ where p is a prime number,  

Zp32 --- a finite field ${\mathbb Z}/p{\mathbb Z}$ where p is less $2^{31}$,  

Z64 --- the ring of integer numbers $z$  such that $-2^{63} \leqslant z < 2^{63}$,  

Q ---  the set of rational numbers, 
 
R --- the set of floating point numbers to store the approximate real numbers with arbitrary mantissa,  

R64 ---  standard floating-point 64-bit numbers (52 digits for mantissa, 11 bits for the order and 1 bit for the sign),   
 
R128 --- standard floating-point 64-bit numbers, equipped with optional 64-bit for the order, 

C --- complexification of R,  

C64 --- complexification of R64, 

C128 --- complexification of R128,   

CZ --- complexification of of Z, 
 
CZp --- complexification of Zp, 

CZp32 --- complexification of Zp32,  

CZ64 --- complexification of Z64,   

CQ --- complexification of Q.  

Examples of simple commutative polynomial rings:
 
SPACE = Z [x,  y,  z]; 

SPACE = R64 [u,  v];  

SPACE = C [x]. 
 
\section{Several numerical sets}
The ring $Z[x,y,z]Z[u,v,w]$, which has two subsets of
 variables, is the polynomial ring with variables $u, v, w$ with coefficients in the polynomial ring $Z[x,y,z]$.

For example, the characteristic polynomial
 of a matrix over the ring $\mathbb{Z}[x,y,z]$ may be obtained as a polynomial with the variable $u$, whose coefficients are
polynomials in the ring $\mathbb{Z}[x,y,z]$.

You can set
algebraic space which defines several numerical sets. For example, the space $C [z] R [x, ~ y] Z [n, ~ m]$ allows to have
 the five names of variables, which defined in the sets $\mathbb {C}$, $ \mathbb {R} $ and $\mathbb {Z} $, respectively.
The first set is the main.  

 
$C[z]R[x, y]Z[n, m]$  
  can be viewed as a polynomial ring of five variables over $\mathbb{C} $, which has the additional properties. 
If the polynomial does not contain the variables $ z $, $ x $, $ y $, then it is a polynomial with coefficients in the set $\mathbb {Z}$.
If the polynomial does not contain the variable $ z$, then it is a polynomial with coefficients in  the set $ \mathbb {R} $.

{\bf  Examples:} 

SPACE=Z[x, y]Z[u]; 

SPACE=R64[u, v]Z[a, b]; 

SPACE=C[x]R[y, z]; 

 %%\section{Group algebras} 

%The definition of the group algebra has the form $KG$, where $K$
 %is a commutative ring of scalars and $G$~--- is a group of
 %noncommutative operators with finite number of generators.
%Names of these generators should begin with capital letters.

%For example, the following group algebras may be defined: 

%$SPACE=Z[x, y]G[U, V]$; (generators U, V),  

%$SPACE=R64[u, v]G[A, B]$; (generators A, B), 

%$SPACE=C[]G[X, Y, Z, T]$; (generators X, Y, Z, T). 

%Each element of such algebra may be considered as a
% sum of terms with functional coefficients.

%$R64[t, y]G[X,~Y,~Z]$~---  is the free group algebra over a function field of two variables $t, y$ over the field  $\mathbb{R}64$
 % with three noncommutative  generators X, Y, Z.
%For example, $A=(t^2+1)X + \sin(t)Y + 3X^2y^3 +(t^2+1)XY^3X^2Y^{-2}X^2$~---is an element of such algebra.

 
\section{Idempotent algebra and tropical mathematics}
User can uses the idempotent algebras.
In this case the signs of "addition"\ and "multiplication"\ for the infix operations can be used for operations in tropical algebra: min, max, addition, multiplication. 

Each numerical sets $\mathbb {R}$, $\mathbb {R}64$, $\mathbb {Z}$ has two additional elements $\infty$ and $-\infty$, and they have different 
elements, which is play the role of zero and unit. We  denote these sets $\hat {\mathbb {R}}$, $\hat {\mathbb {R}}64$, $\hat{\mathbb {Z}}$, correspondingly.
The name of tropical algebra is obtained from three words: (1) a numerical set, (2) an operation, which corresponding to the sign $plus$ 
and (3) an operation, which corresponding to the sign $times$.

The algebras $R64MaxPlus$, $R64MinPlus$, $R64MaxMin$, $R64MinMax$, $ R64MaxMult$, $ R64MinMult$ are defined for the numerical set $\hat {\mathbb {R}}$64. 

{\it RMaxPlus, RMinPlus, RMaxMin, R64MinMax, RMaxMult, RMinMult}  are defined for the numerical set $\hat {\mathbb {R}}$. 

{\it   ZMaxPlus,  ZMinPlus, ZMaxMin, ZMinMax, ZMaxMult,  ZMinMult}  are defined for the numerical set $\hat {\mathbb {Z}}$. 
 
 
For example, for the algebra $ ZMaxPlus $ you can do the following operations.
\smallskip

%begindelete
 {\it \bf Example. }
%enddelete

\vspace*{-3mm}

\begin{verbatim}
SPACE=ZMaxPlus[x, y];
a=2; b=9+x; c=a+b; d=a b+y; \print(c, d)
\end{verbatim}
%begindelete

{\it  \bf The results:} $c = x+9$;  $d = y+2 x+11$.

%enddelete
For each algebra we defined elements {\bf 0} and {\bf 1}, $-\infty$ and $\infty$.
For each element $a$ we defined the operation of closure: $ a^{\times}$, i.e. the amount of $1+a+a^2+a^3+...$.
For the classical algebras this operation is equivalent to $(1-a)^{-1}$. 
 
\section{Constants} 
It is possible to set or replace the following constants.

FLOATPOS --- an amount of decimal positions of the  real number of type R or R64, 
which you can see in the printed form of this number (the default value is 2).

MachineEpsilonR --- machine epsilon for the number of type R and C ( $10^{-29}$ is the  default value).
The number whose absolute value is less than $10^{-29}$, is considered to be a machine zero.
To set the new value of $ 10^{- 30} $, enter the command << MachineEpsilonR64 = 30 >>.

MachineEpsilonR64 --- machine epsilon for the number of type R64 and C64 ( $2^{-36}$ is the  default value).
The number whose absolute value is less than $2^{-36}$, is considered to be a machine zero.
To set the new value of $2^{-48}$, enter the command << MachineEpsilonR64 = 48 >>.

Constant MachineEpsilonR (and MachineEpsilonR64) used in factoring polynomials 
with coefficient of type R (or R64). Each coefficient of the polynomial is divided by the number $ MachineEpsilonR $ 
(or $ MachineEpsilonR64 $) and rounded to integer value.

MOD32 --- --- the module for a finite field of the type Zp32, its value is not greater than $2^{31}$. 
(the default value is 268435399).

MOD --- the module for a finite field of the type Z (the default value is 268 435 399).

RADIAN --- (1/0) is a flag, which indicates that angles are measured in radians (default is 1:  active).

STEPBYSTEP --- (1/0) is a flag, which indicates that you want to display intermediate results (default is 0:  turned off).

EXPAND --- (1/0) is a flag, which indicates that in the input expression all brackets must be disclosed (the default is 1:   active).

SUBSTITUTION --- (1/0) is a flag, which indicates that the names in the input expression must be substituted of their meaning, if they have been defined before (the default is 1:  active).

The constant ACCURACY is an amount of exact decimal positions in the fractional part of a real numbers of type $R$ and $C$ in the result of multiplication or division operation (the default value is 34).If ACCURACY = 100, then the result of arithmetic operation will be rounded to the one hundredth decimal place.  Obviously, the inequality ACCURACY $>$ MachineEpsilonR must hold.

To install the MachineEpsilonR $ 1/10^9 $ (i.e. $1E-9$) enter the command   MachineEpsilonR = 9  .
After that, any number $ a \in R $, will be considered as zero if $ | a | <10^{-9}.$
The value ACCURACY will be set  MachineEpsilonR+5; If you like another value of ACCURACY you can
set the fraction <<MachineEpsilonR=35/49>>. In this case you obtain result: MachineEpsilonR=35 and ACCURACY=49.

For the numbers $ a \in R$64, $ a \in R$128, $ a \in C$64, $ a \in C$128,  there is no constant defining binary places. Arithmetic instructions are used with accuracy that equals 2.22044604925031308e-16.
  However, you can set the number of binary places for MachineEpsilonR64: for example, you can set MachineEpsilonR64=10.

Prime number MOD32 is a characteristic of a finite field.
The constant MOD32 is used when calculations are made in a finite field
Zp32 and it should be less $2^{31}$.

The prime number MOD is also characteristic of the finite field, but it has no restrictions on the absolute value. The constant MOD is used when calculations are made in a finite field Zp.

The constant FLOATPOS determines the number of decimal places,
are printed. In addition, it is used in the factorization of polynomials whose coefficients are 
an approximate numbers of type R or R64. 
Each coefficient of this polynomial is pre-multiplied by the number of $10^{MachineEpsilonR}$ or 
and rounded to an integer value.
But after factoring polynomial extra factor is removed.



\begin{verbatim}
SPACE=Zp32[x, y]; 
MOD32=7; 
f1=37x+42y+55; 
f2=2f1;  
\print(f1, f2 );
\end{verbatim}

